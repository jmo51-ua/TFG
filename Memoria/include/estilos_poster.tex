%%%%%%%%%%%%%%%%%%%%%%%%%%%%%%%%%%%%%%%%%%%%%%%%%%%%%%%%%%%%%%%%%%%%%%%%
% Plantilla Póster TFG/TFM
% Escuela Politécnica Superior de la Universidad de Alicante
% Realizado por: Jose Manuel Requena Plens
% Contacto: info@jmrplens.com / Telegram:@jmrplens
%%%%%%%%%%%%%%%%%%%%%%%%%%%%%%%%%%%%%%%%%%%%%%%%%%%%%%%%%%%%%%%%%%%%%%%%

% EN ESTE ARCHIVO SE DEFINEN LOS CONJUNTOS DE COLORES, ESTILOS DE BLOQUES 
% Y ELEMENTOS GRAFICOS COMO FLECHAS, FORMAS O MODIFICACIONES DE BLOQUES

% ÍNDICE
% 1. Conjunto de colores 
% 2. Estilo de bloques
% 3. Estilo de bloques internos
% 4. Estilo de fondos
% 5. Estilo de notas
% 6. Elementos gráficos
%  6.1. Modificadores de bloques
%  6.2. Flechas



%%%%%%%%%%%%%%%%%%%%%%%% 
% 1. CONJUNTO DE COLORES
%%%%%%%%%%%%%%%%%%%%%%%%
% Estilos disponibles de colores
% 'TFGTFM','Default','Australia','Britain','Sweden','Spain','Russia','Denmark','Germany'
\definecolorstyle{TFGTFM} {}{
	% Colores de fondo
	\colorlet{backgroundcolor}{\colorgrado!20!white} % Fondo general
	\colorlet{framecolor}{black}						 % Color de marco (no utilizado)
	% Colores bloques
	\colorlet{blocktitlebgcolor}{\colorgrado}	% Fondo cabecera
	\colorlet{blocktitlefgcolor}{\colortexto}	% Texto cabecera
	\colorlet{blockbodybgcolor}{white}			% Fondo cuerpo
	\colorlet{blockbodyfgcolor}{black}			% Texto cuerpo
	% Colores de bloques internos
	\colorlet{innerblocktitlebgcolor}{white}				% Borde
	\colorlet{innerblocktitlefgcolor}{black}				% Texto cabecera
	\colorlet{innerblockbodybgcolor}{orange!30!white}	% Fondo
	\colorlet{innerblockbodyfgcolor}{black}				% Texto cuerpo
	% Colores de notas
	\colorlet{notefgcolor}{black}				% Texto
	\colorlet{notebgcolor}{yellow!50!white}		% Fondo
	\colorlet{noteframecolor}{yellow}			% Borde
}

\definecolorstyle{Default}{
    \definecolor{colorOne}{HTML}{DDDDDD}
    \definecolor{colorTwo}{HTML}{0066A8}
    \definecolor{colorThree}{HTML}{FCE565}
}{
    % Colores de fondo
    \colorlet{backgroundcolor}{colorOne}
    \colorlet{framecolor}{colorTwo}
    % Colores bloques
    \colorlet{blocktitlebgcolor}{colorTwo}
    \colorlet{blocktitlefgcolor}{white}
    \colorlet{blockbodybgcolor}{white}
    \colorlet{blockbodyfgcolor}{black}
    % Colores de bloques internos
    \colorlet{innerblocktitlebgcolor}{colorThree}
    \colorlet{innerblocktitlefgcolor}{black}
    \colorlet{innerblockbodybgcolor}{white}
    \colorlet{innerblockbodyfgcolor}{black}
    % Colores de notas
    \colorlet{notefgcolor}{black}
    \colorlet{notebgcolor}{colorThree!70!white}
    \colorlet{notefrcolor}{colorThree}
 }

\definecolorstyle{Australia}{
    \definecolor{colorOne}{HTML}{A2E2C7}
    \definecolor{colorTwo}{HTML}{56555A}
    \definecolor{colorThree}{HTML}{C9AECF}
}{
    % Colores de fondo
    \colorlet{backgroundcolor}{colorOne}
    \colorlet{framecolor}{colorOne!50!colorTwo}
    % Colores bloques
    \colorlet{blocktitlebgcolor}{colorTwo}
    \colorlet{blocktitlefgcolor}{white}
    \colorlet{blockbodybgcolor}{white}
    \colorlet{blockbodyfgcolor}{black}
    % Colores de bloques internos
    \colorlet{innerblocktitlebgcolor}{colorThree}
    \colorlet{innerblocktitlefgcolor}{black}
    \colorlet{innerblockbodybgcolor}{white}
    \colorlet{innerblockbodyfgcolor}{black}
    % Colores de notas
    \colorlet{notefgcolor}{black}
    \colorlet{notebgcolor}{colorThree}
    \colorlet{notefrcolor}{colorThree}
 }

\definecolorstyle{Britain}{
    \definecolor{colorOne}{HTML}{116699}
    \definecolor{colorTwo}{HTML}{CCCCCC}
    \definecolor{colorThree}{HTML}{CC6633}
}{
    % Colores de fondo
    \colorlet{backgroundcolor}{colorOne}
    \colorlet{framecolor}{colorTwo}
    % Colores bloques
    \colorlet{blocktitlebgcolor}{colorTwo}
    \colorlet{blocktitlefgcolor}{colorOne}
    \colorlet{blockbodybgcolor}{white}
    \colorlet{blockbodyfgcolor}{black}
    % Colores de bloques internos
    \colorlet{innerblocktitlebgcolor}{colorThree}
    \colorlet{innerblocktitlefgcolor}{white}
    \colorlet{innerblockbodybgcolor}{white}
    \colorlet{innerblockbodyfgcolor}{black}
    % Colores de notas
    \colorlet{notefgcolor}{black}
    \colorlet{notebgcolor}{colorThree!40!white}
    \colorlet{notefrcolor}{colorThree!60!white}
 }

\definecolorstyle{Sweden}{
    \definecolor{colorOne}{HTML}{116699}
    \definecolor{colorTwo}{HTML}{CCCCCC}
    \definecolor{colorThree}{HTML}{CC6633}
}{
    % Colores de fondo
    \colorlet{backgroundcolor}{colorOne!40!white}
    \colorlet{framecolor}{colorTwo}
    % Colores bloques
    \colorlet{blocktitlebgcolor}{colorTwo!70!black}
    \colorlet{blocktitlefgcolor}{colorOne}
    \colorlet{blockbodybgcolor}{white!90!colorTwo}
    \colorlet{blockbodyfgcolor}{black}
    % Colores de bloques internos
    \colorlet{innerblocktitlebgcolor}{colorThree}
    \colorlet{innerblocktitlefgcolor}{white}
    \colorlet{innerblockbodybgcolor}{white}
    \colorlet{innerblockbodyfgcolor}{black}
    % Colores de notas
    \colorlet{notefgcolor}{black}
    \colorlet{notebgcolor}{colorThree!50!white}
    \colorlet{notefrcolor}{colorThree!50!white}
 }

\definecolorstyle{Spain}{
    \definecolor{colorOne}{HTML}{116699}
    \definecolor{colorTwo}{HTML}{CCCCCC}
    \definecolor{colorThree}{HTML}{CC6633}
}{
    % Colores de fondo
    \colorlet{backgroundcolor}{colorOne!55!white}
    \colorlet{framecolor}{colorTwo}
    % Colores bloques
    \colorlet{blocktitlebgcolor}{colorOne!80!black}
    \colorlet{blocktitlefgcolor}{white}
    \colorlet{blockbodybgcolor}{white}
    \colorlet{blockbodyfgcolor}{black}
    % Colores de bloques internos
    \colorlet{innerblocktitlebgcolor}{colorThree}
    \colorlet{innerblocktitlefgcolor}{white}
    \colorlet{innerblockbodybgcolor}{white}
    \colorlet{innerblockbodyfgcolor}{black}
    % Colores de notas
    \colorlet{notefgcolor}{black}
    \colorlet{notebgcolor}{colorThree!50!white}
    \colorlet{notefrcolor}{colorThree}
 }

\definecolorstyle{Russia}{
    \definecolor{colorOne}{HTML}{116699}
    \definecolor{colorTwo}{HTML}{CCCCCC}
    \definecolor{colorThree}{HTML}{CC6633}
}{
    % Colores de fondo
    \colorlet{backgroundcolor}{white}
    \colorlet{framecolor}{colorOne!50!colorThree!30!}
    % Colores bloques
    \colorlet{blocktitlebgcolor}{colorThree!80!colorTwo!80!black}
    \colorlet{blocktitlefgcolor}{white}
    \colorlet{blockbodybgcolor}{colorTwo!40}
    \colorlet{blockbodyfgcolor}{black}
    % Colores de bloques internos
    \colorlet{innerblocktitlebgcolor}{colorTwo!40}
    \colorlet{innerblocktitlefgcolor}{black}
    \colorlet{innerblockbodybgcolor}{colorTwo}
    \colorlet{innerblockbodyfgcolor}{black}
    % Colores de notas
    \colorlet{notefgcolor}{black}
    \colorlet{notebgcolor}{colorTwo}
    \colorlet{notefrcolor}{colorTwo}
 }
 
\definecolorstyle{Denmark}{
    \definecolor{colorOne}{HTML}{AE0D45}
    \definecolor{colorTwo}{HTML}{7F8897}
    \definecolor{colorThree}{HTML}{C8512D}
}{
    % Colores de fondo
    \colorlet{backgroundcolor}{white}
    \colorlet{framecolor}{white}
    % Colores bloques
    \colorlet{blocktitlebgcolor}{colorTwo}
    \colorlet{blocktitlefgcolor}{colorOne}
    \colorlet{blockbodybgcolor}{white}
    \colorlet{blockbodyfgcolor}{black}
    % Colores de bloques internos
    \colorlet{innerblocktitlebgcolor}{colorThree}
    \colorlet{innerblocktitlefgcolor}{white}
    \colorlet{innerblockbodybgcolor}{white}
    \colorlet{innerblockbodyfgcolor}{black}
    % Colores de notas
    \colorlet{notefgcolor}{black}
    \colorlet{notebgcolor}{colorTwo!50!white}
    \colorlet{notefrcolor}{colorTwo!50!white}
 }

\definecolorstyle{Germany}{
    \definecolor{colorOne}{HTML}{8C7269}
    \definecolor{colorTwo}{HTML}{E89261}
    \definecolor{colorThree}{HTML}{A2C4D9}
}{
    % Colores de fondo
    \colorlet{backgroundcolor}{colorTwo}
    \colorlet{framecolor}{colorThree}
    % Colores bloques
    \colorlet{blocktitlebgcolor}{white}
    \colorlet{blocktitlefgcolor}{colorOne}
    \colorlet{blockbodybgcolor}{white}
    \colorlet{blockbodyfgcolor}{black}
    % Colores de bloques internos
    \colorlet{innerblocktitlebgcolor}{white}
    \colorlet{innerblocktitlefgcolor}{black}
    \colorlet{innerblockbodybgcolor}{colorThree}
    \colorlet{innerblockbodyfgcolor}{black}
    % Colores de notas
    \colorlet{notefgcolor}{black}
    \colorlet{notebgcolor}{colorThree}
    \colorlet{notefrcolor}{colorThree}
 }
 
\definecolorstyle{Qacolor}{
    \definecolor{colorOne}{HTML}{BE0920}
    \definecolor{colorTwo}{HTML}{7F8897}
    \definecolor{colorThree}{HTML}{C8512D}
    \definecolor{colorFour}{HTML}{1E0D85}
}{
    % Colores de fondo
    \colorlet{backgroundcolor}{white}
    \colorlet{framecolor}{white}
    % Colores bloques
    \colorlet{blocktitlebgcolor}{colorTwo!35!white}
    \colorlet{blocktitlefgcolor}{colorOne}
    \colorlet{blockbodybgcolor}{white}
    \colorlet{blockbodyfgcolor}{black}
    % Colores de bloques internos
    \colorlet{innerblocktitlebgcolor}{colorThree}
    \colorlet{innerblocktitlefgcolor}{white}
    \colorlet{innerblockbodybgcolor}{white}
    \colorlet{innerblockbodyfgcolor}{black}
    % Colores de notas
    \colorlet{notefgcolor}{black}
    \colorlet{notebgcolor}{colorFour!15!white}
    \colorlet{notefrcolor}{colorFour!40!white}
 }

\definecolorstyle{Data}{
    \definecolor{colorOne}{HTML}{273746}
    \definecolor{colorTwo}{HTML}{A2D9CE}
    \definecolor{colorThree}{HTML}{C8512D}
    \definecolor{colorFour}{HTML}{AAB7B8}
}{
    % Colores de fondo
    \colorlet{backgroundcolor}{white}
    \colorlet{framecolor}{white}
    % Colores bloques
    \colorlet{blocktitlebgcolor}{colorTwo!35!white}
    \colorlet{blocktitlefgcolor}{colorOne}
    \colorlet{blockbodybgcolor}{white}
    \colorlet{blockbodyfgcolor}{black}
    % Colores de bloques internos
    \colorlet{innerblocktitlebgcolor}{colorThree}
    \colorlet{innerblocktitlefgcolor}{white}
    \colorlet{innerblockbodybgcolor}{white}
    \colorlet{innerblockbodyfgcolor}{black}
    % Colores de notas
    \colorlet{notefgcolor}{black}
    \colorlet{notebgcolor}{colorFour!15!white}
    \colorlet{notefrcolor}{colorFour!40!white}
 }

\definecolorstyle{Utah}{
    \definecolor{colorOne}{HTML}{283747}
    \definecolor{colorOne}{HTML}{AE0D45}
    \definecolor{colorThree}{HTML}{D1F2EB}
    \definecolor{colorFour}{HTML}{AAB7B8}
}{
    % Colores de fondo
    \colorlet{backgroundcolor}{white}
    \colorlet{framecolor}{white}
    % Colores bloques
    \colorlet{blocktitlebgcolor}{colorTwo!35!white}
    \colorlet{blocktitlefgcolor}{colorOne}
    \colorlet{blockbodybgcolor}{white}
    \colorlet{blockbodyfgcolor}{black}
    % Colores de bloques internos
    \colorlet{innerblocktitlebgcolor}{colorThree}
    \colorlet{innerblocktitlefgcolor}{black}
    \colorlet{innerblockbodybgcolor}{white}
    \colorlet{innerblockbodyfgcolor}{black}
    % Colores de notas
    \colorlet{notefgcolor}{black}
    \colorlet{notebgcolor}{colorFour!15!white}
    \colorlet{notefrcolor}{colorFour!40!white}
 }
 
\definecolorstyle{Elena} {
	\definecolor{colorOne}{RGB}{47,104,138}
}{
	% Colores de fondo
	\colorlet{backgroundcolor}{colorOne!20!white} % Fondo general
	\colorlet{framecolor}{black}						 % Color de marco (no utilizado)
	% Colores bloques
	\colorlet{blocktitlebgcolor}{colorOne}	% Fondo cabecera
	\colorlet{blocktitlefgcolor}{white}	% Texto cabecera
	\colorlet{blockbodybgcolor}{white}			% Fondo cuerpo
	\colorlet{blockbodyfgcolor}{black}			% Texto cuerpo
	% Colores de bloques internos
	\colorlet{innerblocktitlebgcolor}{colorOne}				% Borde
	\colorlet{innerblocktitlefgcolor}{white}				% Texto cabecera
	\colorlet{innerblockbodybgcolor}{white}	% Fondo
	\colorlet{innerblockbodyfgcolor}{black}				% Texto cuerpo
	% Colores de notas
	\colorlet{notefgcolor}{black}				% Texto
	\colorlet{notebgcolor}{yellow!50!white}		% Fondo
	\colorlet{noteframecolor}{yellow}			% Borde
}

%%%%%%%%%%%%%%%%%%%%%%%% 
% 2. ESTILO DE BLOQUES. 
%%%%%%%%%%%%%%%%%%%%%%%%
% Estilos disponibles:
% 'TFGTFM', 'Default', 'Basic', 'Minimal','Envelope', 'Corner', 
% 'Slide', 'TornOut', 'Barra',

% Comando para cargar el estilo elegido y guardarlo para otros usos
\newcommand{\estilobloque}[1]{
\useblockstyle{#1}
\def\estilo{#1}
}

\defineblockstyle{TFGTFM}{ % Modificación del estilo 'Slide'
    titlewidthscale=1, bodywidthscale=1, titleleft,
    titleoffsetx=0pt, titleoffsety=0pt, bodyoffsetx=0pt, bodyoffsety=0pt,
    bodyverticalshift=0pt, roundedcorners=0, linewidth=0pt, titleinnersep=1cm,
    bodyinnersep=1cm 
}{
	% Estilo cabecera
    \ifBlockHasTitle 
        \draw[draw=none, left color=blocktitlebgcolor, right color=blocktitlebgcolor!50!blockbodybgcolor]
           (blocktitle.north west) [rounded corners=0] -- (blocktitle.south west) --
        (blocktitle.south east) [rounded corners=5]-- (blocktitle.north east) -- cycle;
    	% Estilo contenido (cuando hay título)
    	\draw[draw=none, fill=blockbodybgcolor] %
        (blockbody.north west) [rounded corners=30] -- (blockbody.south west) --
        (blockbody.south east) [rounded corners=0]-- (blockbody.north east) -- cycle;
    \fi
    % Estilo contenido (haya o no título)
    \draw[draw=none, fill=blockbodybgcolor] %
        (blockbody.north west) [rounded corners=30] -- (blockbody.south west) --
        (blockbody.south east) [rounded corners=30]-- (blockbody.north east) -- cycle;
}

\defineblockstyle{Barra}{
    titlewidthscale=1, bodywidthscale=1, titleleft,
    titleoffsetx=0pt, titleoffsety=0pt, bodyoffsetx=5pt, bodyoffsety=0pt,
    bodyverticalshift=0pt, roundedcorners=0, linewidth=0.2cm,
    titleinnersep=1cm, bodyinnersep=1cm
}{
    \begin{scope}[line width=\blocklinewidth, rounded corners=\blockroundedcorners]
    	% Estilo cabecera (cuando hay título)
       	\ifBlockHasTitle
      		\draw[color=blocktitlefgcolor, line width = 10pt]
               ([xshift=30pt, yshift=5pt]blocktitle.south west) -- ([xshift=-30pt, yshift=5pt]blocktitle.south east);%
    	\else
    		% Estilo cabecera (cuando no hay título)
         	\draw[draw=none]%, fill=blockbodybgcolor]
                 (blockbody.south west) rectangle (blockbody.north east);
        \fi
    \end{scope}
}

%%%%%%%%%%%%%%%%%%%%%%%% 
% 3. ESTILO DE BLOQUES INTERNOS. 
%%%%%%%%%%%%%%%%%%%%%%%%
% Estilos disponibles:
% 'TFGTFM', 'Default', 'Table', 'Basic', 'Minimal', 
% 'Envelope', 'Corner', 'Slide', 'TornOut'

\defineinnerblockstyle{TFGTFM}{
    titlewidthscale=0.25, bodywidthscale=0.75, titlecenter,
    titleoffsetx=0pt, titleoffsety=0pt, bodyoffsetx=0pt, bodyoffsety=0pt,
    bodyverticalshift=0pt, roundedcorners=15, linewidth=3mm,
    titleinnersep=15pt, bodyinnersep=15pt
}{
  % minimum height should be the maximum of \TP@innerblocktitleheight and
  % \TP@innerblockbodyheight
  \node[minimum width=\TP@innerblocktitlewidth, minimum
  height=\TP@innerblockbodyheight, anchor=center] (innerblocktitle) at
  (\TP@innerblockcenter-0.5\TP@innerblockbodywidth+\TP@innerblocktitleoffsetx,
  {-\TP@innerblocktitleheight-0.5\TP@innerblockbodyheight+\TP@innerblocktitleoffsety})
  {};%
  %
  \ifInnerblockHasTitle%
  \node[minimum width=\TP@innerblockbodywidth, minimum
  height=\TP@innerblockbodyheight, anchor=center] (innerblockbody) at
  (\TP@innerblockcenter+0.5\TP@innerblocktitlewidth+\TP@innerblockbodyoffsetx,
  {-\TP@innerblocktitleheight-0.5\TP@innerblockbodyheight+\TP@innerblockbodyoffsety})
  {};%
  %
  \else%
  \node[minimum width=\TP@innerblockbodywidth, minimum
  height=\TP@innerblockbodyheight, anchor=center] (innerblockbody) at
  (\TP@innerblockcenter+\TP@innerblockbodyoffsetx,
  {-\TP@innerblocktitleheight-0.5\TP@innerblockbodyheight}) {};%
  \fi
 \begin{scope}[rounded corners=\innerblockroundedcorners, line width=\innerblocklinewidth]
        \ifInnerblockHasTitle
           % the big rectangle
        \draw[color=innerblocktitlebgcolor, fill=innerblockbodybgcolor]
        (innerblocktitle.north west) rectangle (innerblockbody.south east);%
        \draw[color=innerblocktitlebgcolor] (innerblocktitle.south east) --
        (innerblocktitle.north east); %
        \else
           % No title
           \draw[color=innerblocktitlebgcolor, fill=innerblockbodybgcolor]
               (innerblockbody.south west) rectangle (innerblockbody.north east);
        \fi
    \end{scope}
}


%%%%%%%%%%%%%%%%%%%%%%%% 
% 4. ESTILO DE FONDOS 
%%%%%%%%%%%%%%%%%%%%%%%%
% Estilos disponibles:
% 'TFGTFM', 'Rayos', 'Gradiente', 'GradienteInferior', 'Vacio'

\definebackgroundstyle{TFGTFM}{
    \fill[inner sep=0pt, line width=0pt, color=backgroundcolor]%
    (bottomleft) rectangle (topright);
}

\definebackgroundstyle{Rayos}{
    \draw[line width=0pt, top color=backgroundcolor!70, bottom
    color=backgroundcolor!70!black] (bottomleft) rectangle (topright);
    %
    \begin{scope}
        \foreach \a in {10,20,...,80}{%
            \draw[backgroundcolor, line width=0.15cm](bottomleft) --
            ($(bottomleft)!1!(bottomleft)+(\a:120)$);%
        }
        \foreach \i in {1,2,...,50}{%
            \begin{scope}[shift={($(rand*60,rand*70)$)}]
                \draw[backgroundcolor!50!, line width=0.1cm] (0,0) circle (4);
            \end{scope}
        }
    \end{scope}
}

\definebackgroundstyle{Gradiente}{
    \draw[line width=0pt, bottom color=backgroundcolor, top
     color=backgroundcolor!60!white] (bottomleft) rectangle (topright);
}

\definebackgroundstyle{GradienteInferior}{
    \draw[line width=0pt, bottom color=backgroundcolor!60!white, top
     color=backgroundcolor] (bottomleft) rectangle (topright);
}

\definebackgroundstyle{Vacio}{
  %
}

\definebackgroundstyle{Quadro}{
\shade[upper left=quadro1,upper right=quadro2,lower left=quadro3,lower right=quadro4] (bottomleft) rectangle (topright);
}
%%%
% Comando para incluir imagen de fondo
%%%
\newcommand{\imagenfondo}[2]{
\node[above right,opacity=#1,inner sep=0pt,outer sep=0pt] at (bottomleft) {\includegraphics[width=\paperwidth,height=\paperheight]{#2}};
}


%%%%%%%%%%%%%%%%%%%%%%%% 
% 5. ESTILO DE NOTAS 
%%%%%%%%%%%%%%%%%%%%%%%%
\definenotestyle{Default}{
    targetoffsetx=0pt, targetoffsety=0pt, angle=0, radius=8cm, width=8cm,
    connection=false, rotate=0, roundedcorners=20, linewidth=0pt, innersep=1cm
}{
    \ifNoteHasConnection %% callout note
        \draw[color=notefrcolor, fill=notebgcolor]%
         (notetarget) -- ($(notetarget)!1!4:(notecenter.center)$) --
         ($(notetarget)!1!-4:(notecenter.center)$) --cycle; %
         %
    \fi
    % the body of the note
    \draw[color=notefrcolor, fill=notebgcolor, rounded
    corners=\noteroundedcorners] (notecenter.south west) -- (notecenter.north
    west) -- (notecenter.north east) -- (notecenter.south east) -- cycle;
}

 \definenotestyle{Corner}{
    targetoffsetx=0pt, targetoffsety=0pt, angle=0, radius=8cm, width=12cm,
    connection=false, rotate=0, roundedcorners=20, linewidth=0pt, innersep=1cm
}{
    \ifNoteHasConnection % callout note
      \draw[color=notebgcolor, fill=notebgcolor, drop shadow={shadow
        xshift=0.2cm, shadow yshift=-0.2cm, opacity=0.3}] %
        (notetarget) -- ($(notetarget)!1!4:(notecenter.center)$) --
         ($(notetarget)!1!-4:(notecenter.center)$) --cycle; %
    \fi
    % the body of the note
    % the shape
    \def \border{%
        [rounded corners=0] (notecenter.south west) -- (notecenter.north west) %
        [rounded corners=\noteroundedcorners] -- ($(notecenter.north
        east)-(\noterotate:4.7)$) %
        [rounded corners=\noteroundedcorners] -- ($(notecenter.north
        east)+(-90+\noterotate:1.7)$) %
        [rounded corners=0] -- (notecenter.south east) -- (notecenter.south
        west) -- cycle%
   }
    \fill[color=notebgcolor] \border;
    \coordinate (x) at (\noterotate:1);
    \coordinate (y) at (\noterotate-90:1);
    % the shadow of the corner
    \fill[color=gray,opacity=0.3] ($(notecenter.north east)+3*(y)$) --
        ($(notecenter.north east)+2.5*(y)$) .. %
        controls ($(notecenter.north east)+1.25*(y)$) and ($(notecenter.north
        east)-1.5*(x)+1.25*(y)$) .. %
        ($(notecenter.north east)-1.9*(x)+2.5*(y)$) .. %
        controls ($(notecenter.north east)-4.5*(x)$) .. %
        ($(notecenter.north east)-5.7*(x)$) %
        [rounded corners=\noteroundedcorners] -- ($(notecenter.north east)-4.7*(x)$) %
        [rounded corners=\noteroundedcorners] -- ($(notecenter.north east)+1.7*(y)$) %
        [rounded corners=0] -- ($(notecenter.north east)+3*(y)$);
    % the corner
    \fill[color=notefrcolor] %
        ($(notecenter.north east)+3*(y)$) -- ($(notecenter.north east)+2.5*(y)$) .. %
        controls ($(notecenter.north east)+1.25*(y)$) and ($(notecenter.north
        east)-1.5*(x)+1.25*(y)$) .. %
        ($(notecenter.north east)-1.9*(x)+2.3*(y)$) .. %
        controls ($(notecenter.north east)-4.5*(x)$) .. %
        ($(notecenter.north east)-5.7*(x)$) %
        [rounded corners=\noteroundedcorners] -- ($(notecenter.north east)-4.7*(x)$) %
        [rounded corners=\noteroundedcorners] -- ($(notecenter.north east)+1.7*(y)$) %
        [rounded corners=0] -- ($(notecenter.north east)+3*(y)$);
}

 \definenotestyle{Gradiente}{
    targetoffsetx=0pt, targetoffsety=0pt, angle=0, radius=8cm, width=8cm,
    connection=false, rotate=0, roundedcorners=20, linewidth=1pt, innersep=1cm
}{
    \ifNoteHasConnection % callout note
         % the shadow
         \begin{scope}[opacity=0.3]
            \begin{pgftransparencygroup}
              \coordinate (shadowshift) at (0.2cm,-0.2cm); \fill%
              ($(notetarget)+(shadowshift)$) --
              ($(notetarget)!1!4:(notecenter.center)+(shadowshift)$) --
              ($(notetarget)!1!-4:(notecenter.center)+(shadowshift)$) --cycle; %
              \fill[rounded corners=\noteroundedcorners] %
              ($(notecenter.south west)+(shadowshift)$) -- ($(notecenter.north
              west)+(shadowshift)$) -- ($(notecenter.north east)+(shadowshift)$)
              -- ($(notecenter.south east)+(shadowshift)$) -- cycle;
            \end{pgftransparencygroup}
          \end{scope}
          %% the main drawing
          %
          %% the border
          \draw[color=notefrcolor, line width=\notelinewidth*2]%
          (notetarget) -- ($(notetarget)!1!4:(notecenter.center)$) --
          ($(notetarget)!1!-4:(notecenter.center)$) -- cycle;%
          \draw[color=notefrcolor, line width=\notelinewidth*2, rounded
          corners=\noteroundedcorners]%
          (notecenter.south west) -- (notecenter.north west) --
          (notecenter.north east) -- (notecenter.south east) -- cycle; %
          %
          %% the filling (vertical shading), shared between the note and the connection
          \begin{scope}
            \node[fit=(notetarget)(notecenter.south west)(notecenter.south east)
            (notecenter.north east) (notecenter.north west), inner sep=+0pt]
            (box) {};%
            %
            \clip (notetarget) -- ($(notetarget)!1!4:(notecenter.center)$) --
            ($(notetarget)!1!-4:(notecenter.center)$) -- cycle%
            [rounded corners=\noteroundedcorners] (notecenter.south west) --
            (notecenter.north west) -- (notecenter.north east) --
            (notecenter.south east) -- cycle;
            %
            \draw[draw=none, color=notefrcolor, top color=notebgcolor!60, bottom
            color=notebgcolor] %
            (box.south west) rectangle (box.north east);
          \end{scope}
          %
    \else % the simple note
        \begin{scope}[drop shadow={shadow xshift=0.2cm, shadow yshift=-0.2cm,
           opacity=0.3}]
         \draw[line width=\notelinewidth, rounded corners=\noteroundedcorners,
         color=notefrcolor, top color=notebgcolor!60, bottom color=notebgcolor,
         drop shadow] %
         (notecenter.south west) -- (notecenter.north west) -- (notecenter.north
         east) -- (notecenter.south east) -- cycle;
        \end{scope}
    \fi
}

 \definenotestyle{Sticky}{
    targetoffsetx=0pt, targetoffsety=0pt, angle=0, radius=8cm, width=8cm,
    connection=false, rotate=0, roundedcorners=0, linewidth=0pt, innersep=1cm
}{
    \ifNoteHasConnection %% callout note
    \draw[color=notefrcolor, fill=notebgcolor, drop shadow={shadow
        xshift=0.2cm, shadow yshift=-0.2cm, opacity=0.3}] %
         (notetarget) -- ($(notetarget)!1!4:(notecenter.center)$) --
         ($(notetarget)!1!-4:(notecenter.center)$) --cycle; %
    \fi
    % the body of the note
    % shadow
    \draw[draw=none, fill=gray, opacity=0.3]
        ($(notecenter.north east)+(-0.5,0)$) [rounded corners=40]--%
        (notecenter.north west) [rounded corners=0] -- %
        ($(notecenter.south west)$) .. %
        controls ($0.2*(notecenter.south west) + 0.8*(notecenter.south east)$) .. %
        ($(notecenter.south east)+(-0.2,0.3)$) .. %
        controls ($0.75*(notecenter.south east) + 0.25*(notecenter.north east) - (0.5,0)$) .. %
        ($(notecenter.north east)+(-0.5,0)$);
    % the shape
    \def \border{%
        ($(notecenter.north east)+(-0.5,0)$) [rounded corners=40]--%
        (notecenter.north west) [rounded corners=0] -- %
        ($(notecenter.south west)$) .. %
        controls ($0.2*(notecenter.south west) + 0.8*(notecenter.south east)$) .. %
        ($(notecenter.south east)+(0,0.7)$) .. %
        controls ($0.75*(notecenter.south east) +0.25*(notecenter.north east) -(0.5,0)$) .. %
        ($(notecenter.north east)+(-0.5,0)$)%
    }%
    \draw[color=notefrcolor, fill=notebgcolor]
    \border;
    % the shading in the left top corner
    \begin{scope}
        \clip \border; %
        \begin{scope}[transform canvas={rotate
            around={\noterotate+15:(notecenter.north west)}}]
            \fill[notebgcolor!60!black, path fading=south, opacity=0.6]%
                (notecenter.north west) -- +(-3,0) |- ($(notecenter.north west) + (0,-1.2)$)
                -- ($(notecenter.north west) + (4,-1.2)$) |- ($(notecenter.north west)$);
        \end{scope}
    \end{scope}
}

%%%%%%%%%%%%%%%%%%%%%%%% 
% 6. ELEMENTOS GRÁFICOS
%%%%%%%%%%%%%%%%%%%%%%%%

%%%%%%%%%%%%
% 6.1. Modificadores de bloques
%%%%%%%%%%%%

% Flecha entre cajas
\newcommand{\flechacaja}[3]{
% Rotación de la flecha y altura
\ifthenelse{\equal{#2}{south}}{\def\rotacion{-90},\def\altura{\TP@blockverticalspace}}{
\ifthenelse{\equal{#2}{west}}{\def\rotacion{180},\def\altura{\TP@colspace}}{
\ifthenelse{\equal{#2}{north}}{\def\rotacion{90},\def\altura{\TP@blockverticalspace}}{
\ifthenelse{\equal{#2}{east}}{\def\rotacion{0},\def\altura{\TP@colspace}}
}}}

% Color de la flecha
\ifthenelse{\isempty{#3}}{\def\colorflecha{blockbodybgcolor}}{\def\colorflecha{#3}}

% Estilos de flecha

\ifthenelse{\equal{\estilo}{TFGTFM}}{
\begin{scope}[rotate=\rotacion]
\fill[color=\colorflecha] 
		([shift={(-2mm,-2cm)}]#1.#2) -- 
		([shift={(0.7\altura,0)}]#1.#2) -- 
		([shift={(-2mm,2cm)}]#1.#2)--
		([shift={(-2mm,-2cm)}]#1.#2);
\end{scope}
}

\ifthenelse{\equal{\estilo}{Default}}{
\ifthenelse{\isempty{#3}}{\def\colorflecha{blocktitlebgcolor}}{\def\colorflecha{#3}}
\begin{scope}[rotate=\rotacion]
\fill[color=\colorflecha] 
		([shift={(-2mm,-2cm)}]#1.#2) -- 
		([shift={(0.7\altura,0)}]#1.#2) -- 
		([shift={(-2mm,2cm)}]#1.#2)--
		([shift={(-2mm,-2cm)}]#1.#2);
\end{scope}
}

\ifthenelse{\equal{\estilo}{Basic}}{
\begin{scope}[rotate=\rotacion]
\fill[color=\colorflecha] 
		([shift={(-1.8\blocklinewidth,-2cm)}]#1.#2) -- 
		([shift={(0.7\altura,0)}]#1.#2) -- 
		([shift={(-1.8\blocklinewidth,2cm)}]#1.#2)--
		([shift={(-1.8\blocklinewidth,-2cm)}]#1.#2);
\draw[color=framecolor,line width=\blocklinewidth] 
		([shift={(-1.2\blocklinewidth,-2cm)}]#1.#2) -- 
		([shift={(0.7\altura,0)}]#1.#2) -- 
		([shift={(-1.2\blocklinewidth,2cm)}]#1.#2);
\end{scope}
}

\ifthenelse{\equal{\estilo}{Envelope}}{
\begin{scope}[rotate=\rotacion]
\fill[color=\colorflecha] 
		([shift={(-4\blocklinewidth,-2cm)}]#1.#2) -- 
		([shift={(0.7\altura,0)}]#1.#2) -- 
		([shift={(-4\blocklinewidth,2cm)}]#1.#2)--
		([shift={(-4\blocklinewidth,-2cm)}]#1.#2);
\draw[color=blocktitlebgcolor,line width=\blocklinewidth] 
		([shift={(-3.5\blocklinewidth,-2cm)}]#1.#2) -- 
		([shift={(0.7\altura,0)}]#1.#2) -- 
		([shift={(-3.5\blocklinewidth,2cm)}]#1.#2);
\end{scope}
}

\ifthenelse{\equal{\estilo}{Corner}}{
\begin{scope}[rotate=\rotacion]
\fill[color=\colorflecha] 
		([shift={(-5\blocklinewidth,-2cm)}]#1.#2) -- 
		([shift={(0.7\altura,0)}]#1.#2) -- 
		([shift={(-5\blocklinewidth,2cm)}]#1.#2)--
		([shift={(-5\blocklinewidth,-2cm)}]#1.#2);
\draw[color=blocktitlebgcolor,line width=\blocklinewidth] 
		([shift={(-4.5\blocklinewidth,-2cm)}]#1.#2) -- 
		([shift={(0.7\altura,0)}]#1.#2) -- 
		([shift={(-4.5\blocklinewidth,2cm)}]#1.#2);
\end{scope}
}

\ifthenelse{\equal{\estilo}{Slide}}{
\begin{scope}[rotate=\rotacion]
\fill[color=\colorflecha] 
		([shift={(-2mm,-2cm)}]#1.#2) -- 
		([shift={(0.7\altura,0)}]#1.#2) -- 
		([shift={(-2mm,2cm)}]#1.#2)--
		([shift={(-2mm,-2cm)}]#1.#2);
\end{scope}
}

\ifthenelse{\equal{\estilo}{TornOut}}{
\begin{scope}[rotate=\rotacion]
\fill[color=\colorflecha] 
		([shift={(-8\blocklinewidth,-2cm)}]#1.#2) -- 
		([shift={(0.7\altura,0)}]#1.#2) -- 
		([shift={(-8\blocklinewidth,2cm)}]#1.#2)--
		([shift={(-8\blocklinewidth,-2cm)}]#1.#2);

\end{scope}
}
}

%%%%%%%%%%%%
% 6.2. Flechas
%%%%%%%%%%%%

% Con flecha en ambos extremos
\newcommand{\flechaA}[7]{
\draw [stealth-stealth, line width=12mm,postaction={decorate,decoration={raise=#1,text along path,text align=center,text={|\color{#2}\bfseries|#3}}},#4] (#5) to[#6] (#7);}
% Con flecha solo en el extremo final
\newcommand{\flechaB}[7]{
\draw [-stealth, line width=12mm,postaction={decorate,decoration={raise=#1,text along path,text align=center,text={|\color{#2}\bfseries|#3}}},#4] (#5) to[#6] (#7);}
