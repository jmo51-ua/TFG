%%%%%%%%%%%%%%%%%%%%%%%%%%%%%%%%%%%%%%%%%%%%%%%%%%%%%%%%%%%%%%%%%%%%%%%%
% Plantilla TFG/TFM
% Escuela Politécnica Superior de la Universidad de Alicante
% Realizado por: Jose Manuel Requena Plens
% Contacto: info@jmrplens.com / Telegram:@jmrplens
%%%%%%%%%%%%%%%%%%%%%%%%%%%%%%%%%%%%%%%%%%%%%%%%%%%%%%%%%%%%%%%%%%%%%%%%


%%%%%%%%%%%%%%%%%%%%%%%% 
% CÓDIGO. CONFIGURACIÓN. En el siguiente bloque están los estilos.
%%%%%%%%%%%%%%%%%%%%%%%%
% Paquete para mostrar código de matlab. En caja y lineas numeradas
\usepackage[framed,numbered]{matlab-prettifier}
% Paquete mostrar código de programación de distintos lenguajes
\usepackage{listings}
\lstset{ inputencoding=utf8,
extendedchars=true,
frame=single, % Caja donde se ubica el código
backgroundcolor=\color{gray97}, % Color del fondo de la caja
rulesepcolor=\color{black},
boxpos=c,
abovecaptionskip=-4pt,
aboveskip=12pt,
belowskip=0pt,
lineskip=0pt,
framerule=0pt,
framextopmargin=4pt,
framexbottommargin=4pt,
framexleftmargin=11pt,
framexrightmargin=0pt,
linewidth=\linewidth,
xleftmargin=\parindent,
framesep=0pt,
rulesep=.4pt,
stringstyle=\ttfamily,
showstringspaces = false,
showspaces = false,
showtabs = false,
columns=fullflexible,
basicstyle=\small\ttfamily,
commentstyle=\color{gray45},
keywordstyle=\bfseries,
tabsize=4,
numbers=left,
numbersep=1pt,
numberstyle=\tiny\ttfamily\color{gray75},
numberfirstline = false,
breaklines=true,
postbreak=\mbox{\textcolor{red}{$\hookrightarrow$}\space}, % Flecha al saltar de linea
prebreak=\mbox{\textcolor{red}{$\hookleftarrow$}\space}, % Flecha al saltar de linea
literate=
  {á}{{\'a}}1 {é}{{\'e}}1 {í}{{\'i}}1 {ó}{{\'o}}1 {ú}{{\'u}}1
  {Á}{{\'A}}1 {É}{{\'E}}1 {Í}{{\'I}}1 {Ó}{{\'O}}1 {Ú}{{\'U}}1
  {à}{{\`a}}1 {è}{{\`e}}1 {ì}{{\`i}}1 {ò}{{\`o}}1 {ù}{{\`u}}1
  {À}{{\`A}}1 {È}{{\'E}}1 {Ì}{{\`I}}1 {Ò}{{\`O}}1 {Ù}{{\`U}}1
  {ä}{{\"a}}1 {ë}{{\"e}}1 {ï}{{\"i}}1 {ö}{{\"o}}1 {ü}{{\"u}}1
  {Ä}{{\"A}}1 {Ë}{{\"E}}1 {Ï}{{\"I}}1 {Ö}{{\"O}}1 {Ü}{{\"U}}1
  {â}{{\^a}}1 {ê}{{\^e}}1 {î}{{\^i}}1 {ô}{{\^o}}1 {û}{{\^u}}1
  {Â}{{\^A}}1 {Ê}{{\^E}}1 {Î}{{\^I}}1 {Ô}{{\^O}}1 {Û}{{\^U}}1
  {œ}{{\oe}}1 {Œ}{{\OE}}1 {æ}{{\ae}}1 {Æ}{{\AE}}1 {ß}{{\ss}}1
  {ű}{{\H{u}}}1 {Ű}{{\H{U}}}1 {ő}{{\H{o}}}1 {Ő}{{\H{O}}}1
  {ç}{{\c c}}1 {Ç}{{\c C}}1 {ø}{{\o}}1 {å}{{\r a}}1 {Å}{{\r A}}1
  {€}{{\euro}}1 {£}{{\pounds}}1 {«}{{\guillemotleft}}1
  {»}{{\guillemotright}}1 {ñ}{{\~n}}1 {Ñ}{{\~N}}1 {¿}{{?`}}1,
  }

% Intenta no dividir los códigos en diferentes paginas si es posible
\lstnewenvironment{listing}[1][]
   {\lstset{#1}\pagebreak[0]}{\pagebreak[0]}

% Formato de títulos de los códigos
\DeclareCaptionFont{white}{\color{white}}
\DeclareCaptionFormat{listing}{\colorbox{gray}{\parbox{\textwidth - 2\fboxsep}{#1#2#3}}}
\captionsetup[lstlisting]{format=listing,labelfont=white,textfont=white,font= scriptsize}


%%%%%%%%%%%%%%%%%%%%%%%% 
% CÓDIGO. ESTILOS. Ajústalos a tu gusto
%%%%%%%%%%%%%%%%%%%%%%%%
\lstdefinestyle{Consola}
	{
	basicstyle=\scriptsize\bfseries\ttfamily,
	}
   
\lstdefinestyle{C}
	{
	basicstyle=\scriptsize,
	language=C,
	}
\lstdefinestyle{C-color}
	{
  	breaklines=true,
  	language=C,
  	basicstyle=\scriptsize,
  	keywordstyle=\bfseries\color{green!40!black},
  	commentstyle=\itshape\color{purple!40!black},
  	identifierstyle=\color{blue},
  	stringstyle=\color{orange},
    }
\lstdefinestyle{CSharp}
	{
	basicstyle=\scriptsize,
	language=[Sharp]C,
	escapeinside={(*@}{@*)},
	keywordstyle=\bfseries,
	}
\lstdefinestyle{CSharp-color}
	{
	basicstyle=\scriptsize,
	language=[Sharp]C,
	escapeinside={(*@}{@*)},
	commentstyle=\color{greencomments},
	keywordstyle=\color{bluekeywords}\bfseries,
	stringstyle=\color{redstrings},
	}
\lstdefinestyle{C++}
	{
	basicstyle=\scriptsize,
	language=C++,
 	}
 	
\lstdefinestyle{C++-color}
	{
  	breaklines=true,
  	language=C++,
  	basicstyle=\scriptsize,
  	keywordstyle=\bfseries\color{green!40!black},
  	commentstyle=\itshape\color{purple!40!black},
  	identifierstyle=\color{blue},
  	stringstyle=\color{orange},
    }
\lstdefinestyle{javascript}
	{
	basicstyle=\scriptsize,
	language=javascript,
 	}  
\lstdefinestyle{Javascript-color}
	{
  	breaklines=true,
  	language=Javascript,
  	basicstyle=\scriptsize,
  	keywordstyle=\bfseries\color{green!40!black},
  	commentstyle=\itshape\color{purple!40!black},
  	identifierstyle=\color{blue},
  	stringstyle=\color{orange},
    }  
\lstdefinestyle{PHP}
	{
	basicstyle=\scriptsize,
	language=PHP,
	}
	
\lstdefinestyle{PHP-color}
	{
	basicstyle=\scriptsize,
	language=PHP,
	keywordstyle    = \color{dkblue},
  	stringstyle     = \color{red},
  	identifierstyle = \color{dkgreen},
  	commentstyle    = \color{gray},
  	emph            =[1]{php},
  	emphstyle       =[1]\color{black},
  	emph            =[2]{if,and,or,else},
  	emphstyle       =[2]\color{dkyellow}
  }
  
\lstdefinestyle{Matlab}
	{
	basicstyle=\scriptsize,
	language=Matlab,
	numberstyle=\tiny\ttfamily\color{gray75},
	}
	
\lstdefinestyle{Matlab-color}
	{
	style = Matlab-editor,
	basicstyle=\scriptsize,
	numberstyle=\tiny\ttfamily\color{gray75},
	}
	
\lstdefinestyle{Latex}
	{
	language=[LaTeX]{Tex},
    basicstyle=\scriptsize,
    literate={\$}{{{\bfseries\$}}}1,
    alsoletter={\\,*,\&},
    emph =[1]{\\begin,\\end,\\caption,\\label,\\centering,\\FloatBarrier,
              \\lstinputlisting,\\scalefont,\\addplot,\\input,
              \\legend,\\item,\\subitem,\\includegraphics,\\textwidth,
              \\section,\\subsection,\\subsubsection,\\paragraph,
              \\cite,\\citet,\\citep,\\gls,\\bibliographystyle,\\url,
              \\citet*,\\citep*,\\todo,\\missingfigure,\\footnote},
  	emphstyle =[1]\bfseries,
  	emph = [2]{equation,subequations,eqnarray,figure,subfigure,
  			   condiciones,flalign,tikzpicture,axis,lstlisting,
  			   itemize,description
  			   },
  	emphstyle =[2]\bfseries,
    numbers=none,
	}
	
\lstdefinestyle{Latex-color}
	{
	language=[LaTeX]{Tex},
    basicstyle=\scriptsize,
    commentstyle=\color{dkgreen},
    identifierstyle=\color{black},
    literate={\$}{{{\bfseries\color{Dandelion}\$}}}1, % Colorea el simbolo dollar
    alsoletter={\\,*,\&},
    emph =[1]{\\begin,\\end,\\caption,\\label,\\centering,\\FloatBarrier,
              \\lstinputlisting,\\scalefont,\\addplot,\\input,
              \\legend,\\item,\\subitem,\\includegraphics,\\textwidth,
              \\section,\\subsection,\\subsubsection,\\paragraph,
              \\cite,\\citet,\\citep,\\gls,\\bibliographystyle,\\url,
              \\citet*,\\citep*,\\todo,\\missingfigure,\\footnote},
  	emphstyle =[1]\bfseries\color{RoyalBlue},
  	emph = [2]{equation,subequations,eqnarray,figure,subfigure,
  			   condiciones,flalign,tikzpicture,axis,lstlisting,
  			   itemize,description
  			   },
  	emphstyle =[2]\bfseries,
    numbers=none,
	}
\lstdefinestyle{Java}
{
	basicstyle=\scriptsize,
	language=Java,
}

\lstdefinestyle{Java-color}
{
	basicstyle=\scriptsize,
	language=Java,
  	keywordstyle=\color{blue},
  	commentstyle=\color{dkgreen},
  	stringstyle=\color{mauve},
}
\lstdefinestyle{Python}
{
	language=Python,
	basicstyle=\scriptsize,
	otherkeywords={self},  
	keywordstyle=\bfseries,     
	emphstyle=\bfseries,    
	emph={MyClass,__init__},         
}

\lstdefinestyle{Python-color}
{
	language=Python,
	basicstyle=\scriptsize,
	otherkeywords={self},          
	keywordstyle=\bfseries\color{deepblue},
	emph={MyClass,__init__},         
	emphstyle=\bfseries\color{deepred},    
	stringstyle=\color{deepgreen},
}
\lstdefinestyle{R}
{
	language=R,                     
  	basicstyle=\scriptsize,
  	keywordstyle=\bfseries, 
}
\lstdefinestyle{R-color}
{
	language=R,                     
  	basicstyle=\scriptsize,
  	keywordstyle=\bfseries\color{RoyalBlue}, 
  	commentstyle=\color{YellowGreen},
  	stringstyle=\color{ForestGreen}  
}
