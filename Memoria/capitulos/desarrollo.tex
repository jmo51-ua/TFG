%%%%%%%%%%%%%%%%%%%%%%%%%%%%%%%%%%%%%%%%%%%%%%%%%%%%%%%%%%%%%%%%%%%%%%%%
% Plantilla TFG/TFM
% Escuela Politécnica Superior de la Universidad de Alicante
% Realizado por: Jose Manuel Requena Plens
% Contacto: info@jmrplens.com / Telegram:@jmrplens
%%%%%%%%%%%%%%%%%%%%%%%%%%%%%%%%%%%%%%%%%%%%%%%%%%%%%%%%%%%%%%%%%%%%%%%%

\chapter{Desarrollo}
\label{desarrollo}
\section{Requisitos}
En esta sección realizaremos una toma de los requisitos generales del sistema, seguidos de requisitos específicos.
\subsection{Funcionales}
Los requisitos funcionales son especificaciones que definen las funciones o características que un sistema debe poseer. Estos describen el comportamiento del sistema bajo ciertas condiciones y definen lo que el sistema debe hacer.

\subsubsection{Generales}
\begin{itemize}
    \item \textbf{R1:} Cada uno de los usuarios tendrá acceso a una información específica, pudiendo acceder solamente a los datos y funciones a los que tenga permiso ver.
    
    \item \textbf{R2:} Cada uno de los usuarios tendrá acceso a una información específica, pudiendo acceder solamente a los datos y funciones a los que tenga permiso ver.
    
    \item \textbf{R3:} Se podrán generar informes de rendimiento para los jugadores y el equipo técnico.
    
    \item \textbf{R4:} Cada entrenador de una organización podrá acceder a la información de todos los actores que forman parte de la organización.
    
    \item \textbf{R5:} Los entrenadores tendrán acceso a todas las sesiones de entrenamiento y a estadísticas individuales y grupales de cada jugador.    
\end{itemize}

\subsubsection{Entrenamientos}
\begin{itemize}
    \item \textbf{R6:} Los usuarios tendrán un apartado para acceder a un calendario, que mostrará cada sesión.
    \item \textbf{R7:} Para cada sesión, se mostará específicamente el nombre de ésta, el dia y la hora de inicio y fin de la sesión.
    
\end{itemize}

\subsubsection{Seguimiento y Evaluación del Rendimiento}
\begin{itemize}
    \item \textbf{R8:} Evaluar el rendimiento de los jugadores mediante la recolección de estadísticas durante los entrenamientos y partidos.
    
    \item \textbf{R9:} Mostrar el historial de puntuaciones dadas por el entrenador por cada intervalo de  tiempo definido.
    
    \item \textbf{R9:} Elaborar gráficos de las puntuaciones de los KPIs para todo el equipo para que se pueda apreciar la evolución del equipo en un conjunto.
    
    \item \textbf{R10:} Elaborar gráficos de las puntuaciones de los KPIs para cada jugador, pudiendo apreciarse la evolución individual.
    
    \item \textbf{R11:} Desarrollar un algoritmo estadístico para generar una matriz DAFO (debilidades, Amenazas, Fortalezas y Debilidades) que permitan al entrenador analizar de una manera más precisa las necesidades de sus jugadores.
    
\end{itemize}

\subsubsection{Equipo}
\begin{itemize}
    \item \textbf{R12:} El entrenador podrá ver toda la información detallada del equipo
    
    \item \textbf{R13:} El entrenador podrá ver toda la información detallada de cada jugador individualmente
    
\end{itemize}

\subsubsection{Intervalos de tiempo}
\begin{itemize}
    \item \textbf{R14:} Los usuarios podrán acceder a un histórico de los datos en un intervalo de tiempo correspondiente a una sesión.
    
    \item \textbf{R15:} Los usuarios podrán acceder a un histórico de los datos en un intervalo de tiempo correspondiente a un mes.
    
    \item \textbf{R16:} Los usuarios podrán acceder a un histórico de los datos en un intervalo de tiempo correspondiente a un trimestre.
    
\end{itemize}

\subsubsection{Actores}
\begin{itemize}
    \item \textbf{R17:} Los entrenadores podrán acceder a la información de los actores que componen la organización y sean jugadores.
    
    \item \textbf{R18:} Cada ventana del actor tendrá su información correspondiente al seguimiento de dicho actor.
    
\end{itemize}

\subsection{No Funcionales}
Los requisitos no funcionales describen cómo el sistema debe comportarse. Estos incluyen aspectos como el rendimiento, la seguridad, la usabilidad, la escalabilidad y la disponibilidad del sistema.

\subsubsection{Rendimiento}
\begin{itemize}
    \item \textbf{Capacidad de carga:} La plataforma debe ser capaz de manejar una cantidad considerable de usuarios simultáneos sin afectar al rendimiento.
    
    \item \textbf{Tiempo de Respuesta:} La carga de los datos debe completarse en un tiempo rápido.
    
    \item \textbf{Optimización:} Las consultas a la base de datos deben de estar optimizadas para minimizar el tiempo de espera.
    
\end{itemize}

\subsubsection{Seguridad}
\begin{itemize}
    \item \textbf{Autenticación:} Las operaciones de carga de datos no se realizarán hasta que el usuario no se haya autenticado correctamente.
    
    \item \textbf{Protección de datos:} Los datos personales de los usuarios deben ser almacenados y procesados conforme a las normativas de protección de datos.
    
\end{itemize}

\subsubsection{Usabilidad}
\begin{itemize}
    \item \textbf{Interfaz intuitiva:} La interfaz de usuario debe ser intuitiva y fácil de usar, permitiendo a los usuarios utilizar la plataforma desde cero sin tener que formarse para ello.
    
    \item \textbf{Documentación:} Los usuarios deben poder acceder a una documentación clara para usar la plataforma.
    
\end{itemize}

\subsubsection{Escalabilidad}
\begin{itemize}
    \item \textbf{Escalabilidad horizontal:} La plataforma debe poder escalar horizontalmente (añadiendo mas servidores) para reducir la carga de éstos y mejorar la experiencia del usuario.
    
    \item \textbf{Escalabilidad vertical:} La plataforma debe poder escalar verticalmente (mejorando los servidores) para manejar el aumento de usuarios y la carga de trabajo.
    
\end{itemize}

\subsubsection{Compatibilidad}
\begin{itemize}
    \item \textbf{Multiplataforma:} La plataforma ser accesible en distintos sistemas operativos, así como desde diferentes tipos de dispositivo (móvil u ordenadores).
    
\end{itemize}

\subsubsection{Fiabilidad}
\begin{itemize}
    \item \textbf{Tolerancia a fallos:} El sistema debe manejar los errores de una manera adecuada, proporcionando mensajes de error claros para su rápida resolución.
    
    \item \textbf{Consistencia de datos:} Los datos deben mantenerse consistentes y libres de errores.
    
\end{itemize}

\section{Análisis funcional del sistema}
\subsection{Historias de usuario}

Para la gestión ágil de este proyecto, comenzaremos a describir historias de usuario. Siguiendo una estructura sistemática para reflejar las necesidades de los usuarios que se traducen en requisitos funcionales para el sistema que influirán en el diseño y desarrollo de la plataforma.

%Iniciar Sesión
\begin{tcolorbox}[title= Inicio de sesión]
\textbf{Como} \textit{Usuario},\
\textbf{Quiero} Poder iniciar sesión en la aplicación,\
\textbf{Para que} Pueda acceder a la gestión del equipo y sus estadísticas.
\end{tcolorbox}

\subparagraph{Descripción}
La aplicación debe porporcionar como pantalla inicial un inicio de sesión, donde solicite el correo y la contraseña almacenados en la base de datos. Después serán validados y cargará la siguiente vista.

\subparagraph{Criterios de Aceptación}
\begin{itemize}
\item El usuario debe ver campos para ingresar el correo y la contraseña.
\item Debe haber una lógica de acceso a la base de datos después de pulsar el botón "Iniciar Sesión".
\item Al ingresar credenciales \textbf{válidas}, el usuario debe ser dirigido a la pantalla de selección de equipo.
\item Al ingresar credenciales \textbf{inválidas}, se mostrará el correspondiente cuadro de error, pidiendo volver a introducir las claves.
\end{itemize}
%Seleccionar el Equipo


\subsection{Análisis de entidades de negocio}
aaa
\subsection{Análisis de interfaces}
aaa
\subsection{Análisis de servicios}
aaa
\section{Diseño del sistema}
\subsection{Arquitectura general}
aaa
\subsection{Diseño de interfaces de usuario}
aaa
\subsection{Diseño lógico de la base de datos}
aaa
\subsection{Diseño de servicios}
aaa
\section{Implementación}
\subsection{Arquitectura técnica}
aaa
\subsection{Implementación de interfaces de usuario}
aaa
\subsection{Esquema físico de la base de datos}
aaa
\subsection{Implementación de servicios}
aaa
\subsubsection{Matriz DAFO}
\subsubsection{Promedio GPA}
GPA es un término que se usa mucho en niveles académicos.

\subsubsection{Desviación estándar}
%https://www.udbvirtual.edu.sv/materiales_didacticos/ESA941/clase4.html#:~:text=La%20interpretaci%C3%B3n%20del%20coeficiente%20de,Ejemplo.
aaa
\subsubsection{Pendiente de tendencia}
aaa
\subsubsection{Cambios positivos y negativos entre los intervalos de tiempo}

\section{Pruebas y validación}
aaa
