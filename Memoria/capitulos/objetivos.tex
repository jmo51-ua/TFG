%%%%%%%%%%%%%%%%%%%%%%%%%%%%%%%%%%%%%%%%%%%%%%%%%%%%%%%%%%%%%%%%%%%%%%%%
% Plantilla TFG/TFM
% Escuela Politécnica Superior de la Universidad de Alicante
% Realizado por: Jose Manuel Requena Plens
% Contacto: info@jmrplens.com / Telegram:@jmrplens
%%%%%%%%%%%%%%%%%%%%%%%%%%%%%%%%%%%%%%%%%%%%%%%%%%%%%%%%%%%%%%%%%%%%%%%%

\chapter{Objetivos}
\label{objetivos}

\section{Público objetivo}
La plataforma va dirigida a todos los entrenadores/as de divisiones inferiores con el fin de agilizar y conseguir de la manera más eficaz y eficiente posible el correcto desarrollo de los jugadores.

\section{Desarrollo de talento}
Siguiendo las definiciones de la doctora Deborah Hoare en \cite{DHoare}, después de identificar y seleccionar talentos, es imperativo poner a disposición del deportista una infraestructura adecuada que les permita desarrollar su potencial al máximo. Esto implica no solo un buen programa de entrenamiento y competición bien estructurado, sino también el acceso al equipo necesario para su competencia.

La plataforma facilita este acceso proporcionando un sistema integrado que ayuda a los entrenadores a planificar y ejecutar programas de entrenamiento adaptados a las necesidades individuales o grupales de cada jugador. También permitirá un seguimiento detallado del progreso del equipo, garantizando maximizar sus resultados y que se identifican áreas de mejora comprendidas en una matriz DAFO (Debilidades, Amenazas, Fortalezas y Oportunidades).

Se busca la optimización del proceso de evaluación, implementado herramientas automatizadas que permitan a los entrenadores realizar análisis y valoraciones en tiempo real durante los entrenamientos y partidos. Los datos podrán serán capturasdos mediando dispositivos que el usuario lleve consigo o a traves de aplicaciones móviles.
Los datos recogidos se utilizarán para elaborar análisis históricos de los jugadores para predecir tendencias. Esto ayudará a los entrenadores a tomar decisiones correctas sobre la carga de entrenamiento y la rotación del equipo para optimizar el rendimiento.

En resumen, la plataforma busca ser una herramienta esencial en el arsenal de cualquier entrenador que trabaje con jóvenes jugadores, proporcionándoles las capacidades necesarias para fomentar el desarrollo de sus capacidades en el fútbol y otros deportes. Con esta herramienta, los entrenadores pueden asegurarse de que cada jugador no solo alcanza su potencial técnico y táctico, sino que también recibe el soporte físico y psicológico necesario para su desarrollo integral.