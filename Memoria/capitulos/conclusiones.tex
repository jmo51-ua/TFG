%%%%%%%%%%%%%%%%%%%%%%%%%%%%%%%%%%%%%%%%%%%%%%%%%%%%%%%%%%%%%%%%%%%%%%%%
% Plantilla TFG/TFM
% Escuela Politécnica Superior de la Universidad de Alicante
% Realizado por: Jose Manuel Requena Plens
% Contacto: info@jmrplens.com / Telegram:@jmrplens
%%%%%%%%%%%%%%%%%%%%%%%%%%%%%%%%%%%%%%%%%%%%%%%%%%%%%%%%%%%%%%%%%%%%%%%%

\chapter{Conclusiones}
\label{conclusiones}
En conclusión, podemos decir que la gran mayoría de requisitos han sido satisfechos, consiguiendo así una plataforma de análisis sólida para que entrenadores y coordinadores puedan seguir y evaluar, con mucha mayor precisión, el rendimiento de los jugadores de sus organizaciones. Gracias a este proyecto, he podido adquirir una notable experiencia en el análisis y manejo de datos para la comprensión del usuario. También me ha otorgado la oportunidad de expandir mis conocimientos y habilidades con herramientas que antes no conocía, como Vue.js, draw.io y la herramienta \gls{dao}.

Siendo ésta la primera versión de la plataforma, podemos también deducir su amplio margen de mejora, como por ejemplo la integración de un modelo de inteligencia artificial para que automáticamente lea los datos preparados y genere sus propios gráficos y matrices \gls{dafo}.

Finalmente, habiendo acabado este trabajo puedo destacar y reafirmar la pasión que siento por este grado, poniendo en práctica todos los conocimientos que me han aportado las diferentes asignaturas y profesores.
